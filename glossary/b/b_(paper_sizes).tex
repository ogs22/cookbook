A certain set of paper sizes in common use called the B-series and designated B4, B5, etc. They are defined as follows:
\par
Each paper size is a rectangle with its sides in the ratio
$ 1 : \sqrt{2}. $ B0 has area $ \sqrt{2} \mathrm{m}^2 , $ B1 half of this, 
B2 a quarter, etc. The A- and B- series of paper together form a series
in which each size has $ 1 / \sqrt{2} $ the area of the previous size.
\par
The B sizes are:
\par
B0 = 1000 \ensuremath{ \times } 1414mm;
\par
B1= 707 \ensuremath{ \times } 1000mm;
\par
B2= 500 \ensuremath{ \times } 707mm;
\par
B3= 353 \ensuremath{ \times } 500mm;
\par
B4= 250 \ensuremath{ \times } 353mm;
\par
B5= 176 \ensuremath{ \times } 250mm;
\par
B6= 125 \ensuremath{ \times } 176mm;
\par
B7= 88 \ensuremath{ \times } 125mm;
\par
B8= 62 \ensuremath{ \times } 88mm;
\par
B9= 44 \ensuremath{ \times } 62mm;
\par
B10= 31 \ensuremath{ \times } 44mm.
\par
Paper size B\emph{n} has width $2^{-n/2}$m and height $2^{1/2-n/2}$m.  The sizes are rounded down to ensure that two B4 sheets are slightly smaller than one B3 sheet, etc. 
\par
The height and width of B\emph{n} paper are the geometric means of the heights and widths of A\emph{n}and A\emph{n+1} papers.
  